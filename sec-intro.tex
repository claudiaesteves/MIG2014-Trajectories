\section{Introduction}
\label{sec:intro}

Video Games are constantly displaying larger and more lively virtual environments due to increased computational power and advanced rendering techniques. For example, the recent Grand Theft Auto (GTA) game~\cite{GTA:web} takes place in Los Santos and its surroundings, a completely virtual city. In spite of the impressive quality and liveliness of the scene, Los Santos still remain relatively sparsely populated with virtual people. The reason for this phenomenon is the large computational cost required to get an {\it ambient crowd} in such large environments. To address this issue, the technique of {\it Crowd Patches} has been recently introduced in~\cite{Yersin:2009}. 

{\it Crowd patches} are precomputed elements (patches) of crowd animations. Patches are time-periodic to be endlessly played in time. The boundary conditions of precomputed animations are accurately controlled to enable combining patches in space (i.e., characters can move from in-between patches), and compose large ambient crowds. This technique eases the process of designing an ambient crowd in an efficient way. 

One problem with this technique however, is to compute internal animation trajectories for patches, that satisfy both, time-periodicity and boundary conditions among patches. Satisfying both of these constraints is difficult, because it is equivalent to computing collision-free trajectories that exactly pass through spatio-temporal waypoints (i.e., at some exact position in time) whilst at the same time solving possibly complex situations of interactions with other agents (collision-avoidance). In addition, trajectories should look as natural as possible.

In this paper we propose a new optimization-based method to compute these internal trajectories. Our method starts by assigning linear trajectories which are easy to compute and satisfy both, periodicity and boundary conditions, but at the same time might produce collisions between characters. Then, iteratively, we optimize the trajectories to handle collisions. We try to keep the produced trajectories as close as possible to the initial guessed trajectories, to \note{minimize the magnitude of collision-avoidance maneuvers.} \panayiotis{How do we define this? Do we measure it?} \devin{I don't think we do either.}

To conclude, our main contribution is an optimization-based algorithm to compute high quality animation trajectories (2D global navigation trajectories) for individual crowd patches under constraints (expressed as a set of spatio-temporal boundary control points).

The remainder of this paper is organized as follows: Section \ref{sec:star} proposes a short overview on the state of the art. Section \ref{sec:overview} details our technique to compute these internal trajectories. Then, in Section \ref{sec:results} some results, together with their performance and quality analysis are shown before a brief discussion and concluding remarks. 
